\documentclass[a4paper,12pt]{article}
\usepackage[ukrainian,english]{babel}
\usepackage{ucs}
\usepackage[utf8]{inputenc}
\usepackage[T2A]{fontenc}
\usepackage{amsmath}
\usepackage{amsfonts}
\usepackage{graphicx}
\usepackage{tabto}
\usepackage[left=20mm, top=20mm, right=20mm, bottom=20mm, nohead, nofoot]{geometry}
\usepackage{hyperref}
\usepackage{tgbonum}

\begin{document}

\begin{titlepage}
		\begin{center}
		$\newline$
		\vspace{5cm}
		$\newline$
		{\LARGE\textbf{Лабораторна робота №1\\з предмету\\Теоретико числові алгоритми}}
		\vspace{10cm}
		\begin{flushright}
			\textbf{Роботу виконала:}\\Бекешева Анастасія\\3-го курсу\\групи ФІ-12
		\end{flushright}
		\vspace{2cm}
		\begin{flushright}
			\textbf{Приймав:}\\Якимчук Олексій
		\end{flushright}
	\end{center}
\end{titlepage}
\newpage
\section{Мета.}
\tab Практичне ознайомлення з рiзними методами факторизацiї чисел, реалiзацiя цих методiв i їх п
рiвняння. Видiлення переваг, недолiкiв та особливостей застосування алгоритмiв факторизацiї. Застосування комбiнацiї алгоритмiв факторизацiї для пошуку канонiчного розкладу заданого числа.
\section{Постановка задачi та варiант завдання.}
\tab Пошук канонiчного розкладу великого числа, використовуючи вiдомi методи факторизацiї, а також особлива увага алгоритму Брілхарта-Моррісона. Варіант 1.
\section{Хiд роботи.}
\subsection{План.}
\begin{enumerate}
	\item Створення  \href{https://github.com/nastyabekesheva/NTA-labs/tree/main/lab-1}{github repo}. 
	\item Імплементація імовірнісного тесту Соловея-Штрассена.
	\item Імплементація методу проблих ділень.
	\item Імплементація $\rho$-методу Полларду.
	\item Імплементація (багатомученна) Бріллхарта-Моррісона.
	\item Підняття Docker.
\end{enumerate}
\subsection{Проблеми.}
\tab В БМ були проблеми з генеруванням гладних чисел, бо в ідеалі треба ланцюговий дріб генерувати поки не вистачить гладких чисел, а в методі ну трохи не так описано, звісно до цього я могла і сама догадатись, але щоб студенти в майбутньому менше мучались додайте цей пунктик в методу будь ласка :)
 Ну і власне я фіксила це вже поверх купи написаного коду, тому воно костильне, але як мінімум на числах з дз працює. 
\section{Результати}

\tab\texttt{1495056764861639599 = 17 * 6871 * 853103 * 15003319\\
Elapsed time: 32.2915358543396 seconds}.\\

\texttt{15196946347083 = 3 * 89 * 2297 * 24779017\\
Elapsed time: 4.913546800613403 seconds}.\\


\texttt{61333127792637 = 3 * 89 * 2297 * 100005263\\
Elapsed time: 6.326192855834961 seconds}.\\

\texttt{17873 = 61 * 293\\
Elapsed time: 0.12788724899291992 seconds}.

\newpage
\section{Висновки}
\tab В результаті виконання лабораторної роботи повторно ознайомилась з матеріалами з лекцій. Моя реалізація БМ не є дуже дієвою на великих числах через вище описану проблему. Але можу зазначити, що вийшло реалізувати швидкий розв'язок СЛР. В випадку з числами які мають декілька простих не дуже великих дільників комбінація методу пробних ділень та  $\rho$-методу Полларда справляються непогано.


\end{document}